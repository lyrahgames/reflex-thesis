\documentclass[crop=false]{standalone}
\usepackage{standard}
\begin{document}
  \newgeometry{top=0mm,bottom=0mm,right=38mm,left=38mm}
  {
    \thispagestyle{empty}
    \null
    \vfill
    \hrule
    \medskip
    \begin{abstract}
      \itshape
      The smoothing of surface curves is an essential tool in mesh processing to allow applications, such as surgical planning, to segment, edit, and cut surfaces.
      As curves are typically designed by domain experts to mark relevant surface regions, its smoothed counterpart should retain close to the original and easily adjustable by the user.
      Previous solutions that are based on energy-minimizing splines or generalizations of Bézier splines need a large number of control points to achieve this behavior and may suffer from poor performance or numerical instabilities.
      This thesis aims for the development and a detailed implementation strategy of an
      % efficient and robust
      algorithm for smoothing discrete curves on triangular surface meshes.
      The method is based on the optimization of local geodesic curvatures to obtain results close to the initial curve.
      % and achieves real-time performance through parallelization on the CPU and GPU.
      The implementation uses the C++ programming language
      % combined with the OpenGL graphics API
      and is fully provided as an open-source code repository.
      % Using appropriate benchmarks and the \citetitle{thingi10k} dataset, the efficiency and robustness of the algorithm is evaluated.
      % Finally, the curve smoothing approach is applied in a medical context to the automatic segmentation of lung lobes.
    \end{abstract}
    \medskip
    \hrule
    \vfill
  }
  \restoregeometry
\end{document}
