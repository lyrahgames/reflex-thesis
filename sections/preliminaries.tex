\documentclass{stdlocal}
\begin{document}
\section{Preliminaries} % (fold)
\label{sec:preliminaries}

To systematically approach the design and implementation of curve smoothing algorithms for surface meshes, basic knowledge in the topic of differential geometry of curves and its application and generalization to polyhedral surfaces and surface mesh curves is administrable.
It allows to rigorously formulate the initial and boundary value problems for discrete geodesics which build the foundation of curve smoothing as geodesics can be interpreted as the class of smoothest curves.
Alas, differential geometry heavily builds on the mathematical tools given in the topic of analysis on manifolds and, as a consequence, the reader is assumed to be familiar with its basic concepts.
For convenience, section~\ref{sec:analysis_on_manifolds} in the appendix still provides a brief introduction to the topic to remind of and clarify notations.
% In the following, a brief overview of the main concepts is given.

\subsection{Differential Geometry of Curves} % (fold)
\label{sub:differential_geometry}

  The theory of smooth curves in space or smooth surfaces is one of the main concerns of classical differential geometry.
  Therefore, we refer to some standard textbooks, namely \textcite{goldhorn2009}, \textcite{carmo2016}, \textcite{kuehnel2013}, and \textcite{stahl2013}, for a more excessive and thorough introduction on this topic.
  For the purpose of consistent notation and as a reminder, the main concepts of smooth curves are briefly given in the following in the sense of classical and modern differential geometry.

  In the field of analysis, it is a natural approach to extend a well-founded and -understood theory for smooth objects to their discrete counterparts and vice versa.
  Hereby, discrete and smooth objects will be typically characterized by the limit of sequences of smooth and discrete objects, respectively.
  This procedure may then allow for a generalization of properties and statements from one to the other case.
  Surface mesh curves, as they will be defined in section~\ref{sub:polyhedral_surfaces}, can exactly be seen as such discrete counterparts to the shapes of special classes of smooth parameterized curves.

  \begin{definition}[Parameterized Curves]
    Let $n\in\setNatural$, $k\in\setNatural_{\infty}$, and $[a,b]\subset\setReal$ be a compact interval.
    Then a $k$-times continuously differentiable function $\function{γ}{[a,b]}{\setReal^n}$ is called an ($n$-dimensional) (parameterized) curve (of class $\mathrm{C}^k$).

    For $k=\infty$, γ is also called a smooth ($n$-dimensional) (parameterized) curve.
    \begin{itemize}
      \item $\text{γ is closed} :\iff γ(a) = γ(b)$
      \item $\text{γ is simple} :\iff \forall t\in[a,b)\colon \forall s\in(t,b],(t,s)\neq(a,b)\colon\quad γ(t)\neq γ(s)$
      \item $\text{γ is regular} :\iff \forall t\in (a,b)\colon\quad γ'(t)\neq 0$
      \item $\text{γ is parameterized by arc length} :\iff \norm{γ'} = 1$
    \end{itemize}
    % For brevity, a parameterized curve will be called curve and the curve's dimensionality will be left out if the context is clear about or not requiring those signifiers.
  \end{definition}
  For most applications, the actual parameterization of a curve is not essential.
  Only its shape, which is given by the image $γ([a,b])$, is used and referred to.
  But using only the shape and neglecting its parameterization does only make sense if the shape at least fulfills the properties of a differentiable 1-manifold.
  Unfortunately, the shape of parameterized curves exhibits more general structures and, thus, does not necessarily abide to these constraints.
  As a consequence, more specialized classes of curves need to be looked at.

  \begin{figure}[h]
    \centering
    \begin{subfigure}[t]{0.32\linewidth}
      \center
      \includegraphics[width=\linewidth]{plots/curve-example-1.pdf}
      \caption{%
        Zero-Velocity Curve
        \[
          \begin{aligned}[t]
            &\function{γ}{[-1,1]}{\setReal^2} \\
            &γ(t) \define
            \begin{pmatrix}
              t^3 \\
              2t^2
            \end{pmatrix}
          \end{aligned}
        \]
      }
    \end{subfigure}
    \begin{subfigure}[t]{0.32\linewidth}
      \center
      \includegraphics[width=\linewidth]{plots/curve-example-3.pdf}
      \caption{%
        Lissajous Curve
        \[
          \begin{aligned}[t]
            &\function{γ}{[0,2π]}{\setReal^2} \\
            &γ(t) \define
            \begin{pmatrix}
              \cos t \\
              \cos\roundBrackets{\frac{3}{2}t}
            \end{pmatrix}
          \end{aligned}
        \]
      }
    \end{subfigure}
    \begin{subfigure}[t]{0.32\linewidth}
      \center
      \includegraphics[width=\linewidth]{plots/curve-example-5.pdf}
      \caption{%
        Half-Circle
        \[
          \begin{aligned}[t]
            &\function{γ}{[0,π]}{\setReal^2} \\
            &γ(t) \define
            \begin{pmatrix}
              \cos t \\
              \sin t
            \end{pmatrix}
          \end{aligned}
        \]
      }
    \end{subfigure}

    \begin{subfigure}[t]{0.32\linewidth}
      \center
      \includegraphics[width=\linewidth]{plots/curve-example-2.pdf}
      \caption{%
        Teardrop Curve
        \[
          \begin{aligned}[t]
            &\function{γ}{[0,2π]}{\setReal^2}\\
            &γ(t) \define
            \begin{pmatrix}
              \cos(t) \\
              \sin(t)\sin^3\roundBrackets{\frac{1}{2}t}
            \end{pmatrix}
          \end{aligned}
        \]
      }
    \end{subfigure}
    \begin{subfigure}[t]{0.32\linewidth}
      \center
      \includegraphics[width=\linewidth]{plots/curve-example-4.pdf}
      \caption{%
        Lissajous Curve
        \[
          \begin{aligned}[t]
            &\function{γ}{[0,2π]}{\setReal^2} \\
            &γ(t)\define
            \begin{pmatrix}
              \cos t \\
              \frac{1}{2} \sin 2t
            \end{pmatrix}
          \end{aligned}
        \]
      }
    \end{subfigure}
    \begin{subfigure}[t]{0.32\linewidth}
      \center
      \includegraphics[width=\linewidth]{plots/curve-example-6.pdf}
      \caption{%
        Circle
        \[
          \begin{aligned}[t]
            &\function{γ}{[0,2π]}{\setReal^2} \\
            &γ(t)\define
              \begin{pmatrix}
                \cos t \\
                \sin t
              \end{pmatrix}
          \end{aligned}
        \]
      }
    \end{subfigure}
    \caption[Examples of Parameterized Curves]{%
      \textbf{Examples of Parameterized Curves}\\
      All the plots show a smooth two-dimensional parameterized curve γ on a compact interval with different additional properties.
      The first row from left to right shows no special properties, a regular curve with self-intersection, and a regular and simple curve parameterized by arc length.
      The second row shows their closed counterparts.
    }
  \end{figure}

  The derivative of a regular curve is never zero and therefore a map of full-rank at every point.
  Hence, a regular curve is an immersion into space and its shape is an immersed submanifold.
  For some applications, a regular curve is therefore already sufficiently useful.
  A curve parameterized by arc length is hereby only a technical way of defining a canonical parameterization for regular curves that will be used to simplify the definition of curvature.
  It is clear by definition, that every curve parameterized by arc length is also a regular curve.

  To also make a regular curve an embedding, it needs to be injective.
  To state that a curve is simple, ensures there are self-intersections and makes it an injective function.
  Then by application of the theorem, the image of the embedding is an embedded 1-manifold.

  \begin{corollary}
    Let $n\in\setNatural$ and $k\in\setNatural_\infty$.
    Then a simple and regular parameterized curve $\function{γ}{[a,b]}{\setReal^n}$ of class $\mathrm{C}^k$ is an embedding of class $\mathrm{C}^k$.
    Hence, $γ([a,b])$ is a 1-manifold of class $\mathrm{C}^k$ embedded in $\setReal^n$.
  \end{corollary}

  Every curve that is parameterized by arc length is automatically a regular curve.
  A regular parameterized curve is by definition a $\mathrm{C}^1$-embedding into $\setReal^n$.
  As a consequence, the image of a regular curve is a 1-manifold of class $\mathrm{C}^1$ embedded in $\setReal^n$.
  The statement that a curve is simple makes sure that other intersections than the first with the last points are not allowed to occur.
  A closed curve can therefore still be simple.

  \begin{definition}[Length of Curves]
    Let $n\in\setNatural$, $k\in\setNatural_{\infty}$, $[a,b]\subset\setReal$ be a compact interval, and $\function{γ}{[a,b]}{\setReal^n}$ be an $n$-dimensional parameterized curve of class $\mathrm{C}^k$.
    Then the length of γ is defined as follows.
    \[
      L(γ) \define \integral{a}{b}{\norm{γ'(t)}}{t}
    \]
    For convenience, we also define the arc length of γ by the following function.
    \[
      \function{s}{[a,b]}{[0,L(γ)]},\quad s_γ(t) \define \integral{a}{t}{\norm{γ'(x)}}{x}
    \]
  \end{definition}
  The function $\norm{γ'}$ is continuous on $[a,b]$ and, consequently, uniformly continuous.
  This means that $L(γ)$ is well-defined for all parameterized curves.
  According to the fundamental theorem of calculus, $s_γ$ is continuously differentiable with derivative $s_γ' = \norm{γ'}$ and surjective.
  For curves parameterized by arc length, it holds that $L(γ)=b-a$.

  \begin{lemma}[Regular curves can be parameterized by arc length]
    Let $n\in\setNatural$, $k\in\setNatural_{\infty}$, $[a,b]\subset\setReal$ be a compact interval, and $\function{γ}{[a,b]}{\setReal^n}$ be an $n$-dimensional parameterized curve of class $\mathrm{C}^k$ that is regular.
    Then up to a constant shift, there exists a unique $k$-times continuously differentiable and bijective function $\function{u}{[c,d]}{[a,b]}$ on a compact interval $[c,d]\subset\setReal$ with strictly positive derivative, such that the composition $γ\composition u$ is a curve parameterized by arc length.
  \end{lemma}
  \begin{proof}
    The restriction of $\norm{\cdot}$ to $\setReal^n\setminus\set{0}{}$ is an infinitely differentiable function.
    The regularity of γ implies that $\norm{γ'}$ and, as a direct consequence, the derivative of $s_γ$ are strictly positive.
    Hence, $s_γ$ is bijective with $k$-times continuously differentiable inverse.
    The derivative of $s_γ^{-1}$ must also be strictly positive and continuous as a composition of continuous functions.
    \[
      \roundBrackets{s_γ^{-1}}' = \frac{1}{s_γ'\composition s_γ^{-1}} = \frac{1}{\norm{γ'}\composition s_γ^{-1}} > 0
    \]
    Define $\tilde{γ}\define γ\composition s_γ^{-1}$.
    Then by differential calculus, it follows, that $\tilde{γ}$ is $k$-times continuously differentiable and parameterized by arc length.
    \[
      \norm{\tilde{γ}'}
      = \norm{γ'\composition s_γ^{-1} \cdot \roundBrackets{s_γ^{-1}}'}
      = \norm{\frac{γ'\composition s_γ^{-1}}{s_γ'\composition s_γ^{-1}}}
      = \norm{\frac{γ'\composition s_γ^{-1}}{\norm{γ'}\composition s_γ^{-1}}}
      = \frac{\norm{γ'\composition s_γ^{-1}}}{\norm{γ'\composition s_γ^{-1}}}
      = 1
    \]
    This shows that $s_γ^{-1}$ fulfills the required properties and the existence of a parameterization by arc length.
    To show its uniqueness up to a constant shift, assume an arbitrary function $u$ as given in the lemma.
    Applying the same reasoning as for $s_γ$, it is clear that $\inverse{u}$ also is $k$-times continuously differentiable with strictly positive derivative.
    By calculation, we may then show its connection to the arc length.
    \[
      1
      = \norm{(γ\composition u)'}
      = \norm{γ'\composition u} \cdot u'
      = \frac{\norm{γ'\composition u}}{\roundBrackets{\inverse{u}}'\composition u}
      = \frac{\norm{γ'}}{\roundBrackets{\inverse{u}}'}
      \quad\implies\quad
      \norm{γ'} = \roundBrackets{\inverse{u}}'
    \]
    To integrate this formula, let $t\in[a,b]$ be arbitrary and use again the fundamental theorem of calculus.
    \[
      s_γ(t)
      = \integral{a}{t}{\norm{γ'(x)}}{x}
      = \integral{a}{t}{\roundBrackets{\inverse{u}}'(x)}{x}
      = \inverse{u}(t) - \inverse{u}(a)
      = \inverse{u}(t) - c
    \]
    \[
      u = s_γ^{-1}\composition s_γ\composition u
      = s_γ^{-1}(\cdot - c)\composition \inverse{u}\composition u
      = s_γ^{-1}\roundBrackets{\cdot - c}
    \]
    This shows that, up to a constant shift, $u$ is the same as $s_γ^{-1}$ and the uniqueness of the parameterization by arc length.
  \end{proof}

  \begin{definition}[Canonical Parameterization by Arc Length]
    Let γ be a regular curve.
    Then its canonical parameterization by arc length $\bar{γ}$ is defined by the following expression.
    \[
      \bar{γ} \define γ\composition s_γ^{-1}
    \]
  \end{definition}

  This lemma shows that, in the sense of regular curves, it is sufficient to handle curves that are parameterized by arc length.
  Only for regular curves, it is possible to talk about their curvature.

  \begin{definition}[Curvature of Regular Curves]
    Let γ be a curve parameterized by arc length and φ be a regular curve.
    Their respective curvatures $κ(γ)$ and $κ(φ)$ are defined by the following expressions.
    \[
      κ(γ)\define\norm{γ''}
      \separate
      κ(φ)\define κ(\bar{φ})\composition s_φ
    \]
  \end{definition}

  The surfaces, we are looking at, are given by compact orientable Riemannian submanifolds (with boundary) of dimension two embedded in $\setReal^3$.

  \begin{definition}[Geodesic and Normal Curvature of Curves]
    Let $M$ be a $k$-dimensional Riemannian submanifold embedded in $\setReal^n$ and $\function{γ}{[a,b]}{M}$ a curve parameterized by arc length.
    The geodesic curvature $κ_\mathrm{g}(γ)$ and the normal curvature $κ_\mathrm{n}$ of γ arg given by the following expressions.
    \[
      κ_\mathrm{g}(γ)(t) = \norm{P_{\mathrm{T}_{γ(t)}M} γ''(t)}
      \separate
      κ_\mathrm{n}(γ)(t) = \norm{P_{\mathrm{N}_{γ(t)}M} γ''(t)}
    \]
  \end{definition}

  \begin{corollary}
    Let $M$ be a $k$-dimensional Riemannian submanifold embedded in $\setReal^n$ and $\function{γ}{[a,b]}{M}$ a curve parameterized by arc length.
    \[
      κ^2(γ) = κ_\mathrm{g}^2(γ) + κ_\mathrm{n}^2(γ)
    \]
  \end{corollary}

  \begin{definition}[Oriented Curvatures of Curves]
    Let $M$ be a two-dimensional oriented Riemannian submanifold embedded in $\setReal^3$ and $\function{γ}{[a,b]}{M}$ a curve parameterized by arc length.
    Let $N$ be a field of positively oriented unit normals on $M$ and $T$ be a field of tangent vectors such that for all $t\in[a,b]$, $(γ'(t),T(t))$ is positively oriented basis in $\mathrm{T}_{γ(t)}M$.
    \[
      κ_\mathrm{g}(γ) = \scalarProduct{T}{γ''}
      \separate
      κ_\mathrm{n}(γ) = \scalarProduct{N}{γ''}
    \]
  \end{definition}

  \begin{definition}[Geodesics]
    Let $M$ be a two-dimensional Riemannian submanifold embedded in $\setReal^3$ and $\function{γ}{[a,b]}{M}$ a curve parameterized by arc length.
    The curve γ is called a geodesic if $κ_g(γ) = 0$.
  \end{definition}

  \begin{definition}[Geodesics Problems]
    Let $M$ be a two-dimensional Riemannian submanifold embedded in $\setReal^3$.

    The problem of finding a geodesic $\function{γ}{[a,b]}{M}$ for a given starting point $γ(a)$ and a given direction $γ'(a)$ is called the geodesic initial value problem.

    The problem of finding a geodesic $\function{γ}{[a,b]}{M}$ for given start and end points $γ(a)$ and $γ(b)$ is called the geodesic boundary value problem.
  \end{definition}

% subsection differential_geometry (end)

\subsection{Polyhedral Surfaces} % (fold)
\label{sub:polyhedral_surfaces}

  There are definitions for manifolds with corners.
  In the topological case, these are equivalent to manifolds with boundaries.
  For the smooth case, it is different.
  The half space is only homeomorphic and not diffeomorphic to the cube space.
  so, we could generalize a polyhedral surface as a smooth manifold with corners.
  on the definitions, there is no uniform agree.
  So, we will not use the generalization of manifolds with corners.
  we will use the theory of polthier which should be applicable to manifolds with corners.

  \begin{definition}[Triangle]
    Let $n\in\setNatural$ with $n\geq 2$ and $A,B,C\in\setReal^n$ affinely independent points.
    Then the (topological) triangle $\triangle$ (embedded in $\setReal^n$) is given by the following expression.
    \[
      \triangle \define \set{uA + vB + wC}{u,v,w\in[0,1], u+v+w=1}
    \]
    The triangle's vertices $\mathscr{V}(\triangle)$ and edges $\mathscr{E}(\triangle)$ are hereby defined as follows.
    \[
      \mathscr{V}(\triangle) \define \set{A,B,C}{}
      \separate
      \mathscr{E}(\triangle) \define \set{\overline{AB}, \overline{BC}, \overline{CA}}{}
    \]
    \[
      \partial\triangle \define \textstyle\bigcup\mathscr{E}(\triangle)
    \]
    \[
      Π \define \set{\function{π}{\set{1,2,3}{}}{\mathscr{V}(\triangle)}}{\text{π bijective}}
    \]
    \[
      D \define \set{(u,v) \in [0,1]^2}{u + v < 1}
    \]
    \[
      \function{φ_π}{\overline{D}}{\triangle}
      \separate
      φ_π(u,v) \define (1 - u - v) π(1) + u π(2) + v π(3)
    \]
    \[
      π\sim π' :\iff \exists σ\in\mathrm{S}_3, \mathrm{sgn}\, σ = 1 \colon π' = π\circ σ
    \]
    \[
      Π/\sim = \set{[π], [\bar{π}]}{}
    \]
    \[
      \triangle_{[π]} \define (\triangle, [π])
    \]
    \[
      \mathscr{E}(\triangle_{[π]}) \define \set{\overrightarrow{π_1π_2},\overrightarrow{π_2π_3}, \overrightarrow{π_3π_1}}{}
    \]
    \[
      ϑ_\triangle(A) = \arccos\frac{\scalarProduct{\overrightarrow{AB}}{\overrightarrow{AC}}}{\norm{\overrightarrow{AB}}\norm{\overrightarrow{AC}}}
    \]
  \end{definition}
  A triangle is an orientable topological 2-manifold with boundary embedded in $\setReal^n$.
  The atlas is given by $\set{\roundBrackets{φ_π(D),φ_π|_D^{-1}}}{π\in Π}$
  The inner set $\triangle^\circ$ is an orientable smooth 2-manifold without boundary.
  Its atlas is given by the restrictions $(\triangle^\circ, φ_π|_{D^\circ}^{-1})$.
  Triangles are planar surfaces and, hence, geodesics are given by straight lines.
  Furthermore, the triangle is by definition a convex set and therefore always provides a unique solution for the geodesic boundary value problem.

  \begin{definition}[Polyhedral Surface and Surface Mesh]
    Let $n\in\setNatural$ with $n\geq 2$ and $\mathscr{T}\neq\emptyset$ be a finite set of triangles embedded in $\setReal^n$.
    Let $S=\bigcup\mathscr{T}$ be a two-dimensional topological manifold (with boundary), such that for all $\triangle_1,\triangle_2\in\mathscr{T}$ with $\triangle_1\neq\triangle_2$ the following holds.
    \[
      \triangle_1\cap\triangle_2 \in \set{\emptyset}{} \cup [\mathscr{V}(\triangle_1)\cap\mathscr{V}(\triangle_2)] \cup [\mathscr{E}(\triangle_1)\cap\mathscr{E}(\triangle_2)]
    \]
    In this case, $S$ is called a (topological) polyhedral surface (embedded in $\setReal^n$).
    With $\mathscr{V}(S)$ we denote its vertices, with $\mathscr{E}(S)$ its edges, and with $\mathscr{F}(S)$ its faces.
    \[
      \mathscr{V}(S) \define \bigcup_{\triangle\in\mathscr{T}} \mathscr{V}(\triangle)
      \separate
      \mathscr{E}(S) \define \bigcup_{\triangle\in\mathscr{T}} \mathscr{E}(\triangle)
      \separate
      \mathscr{F}(S) \define \mathscr{T}
    \]
    (topological) surface mesh.
    \[
      \mathscr{M}(S) \define \textstyle\bigcup \mathscr{E}(S)
    \]
    \[
      \partial\mathscr{E}(S) \define \set{e\in\mathscr{E}(S)}{\exists!\triangle\in\mathscr{F}(S)\colon e\subset\triangle}
      \separate
      \mathscr{E}^\circ(S) \define \mathscr{E}(S) \setminus \partial\mathscr{E}(S)
    \]
    \[
      \partial S \define \textstyle\bigcup \partial\mathscr{E}(S)
      \separate
      \partial\mathscr{V}(S) \define \mathscr{V}(S) \cap \partial S
      \separate
      \mathscr{V}^\circ(S) \define \mathscr{V}(S) \setminus \partial\mathscr{V}(S)
    \]
  \end{definition}
  A surface mesh is again a topological 2-manifold.

  \begin{definition}[Oriented Surface Mesh]
    Let $S$ be a surface mesh and $\function{Π}{\mathscr{F}(S)}{\bigsqcup_{\triangle\in\mathscr{F}(S)} Π_\triangle}$.
    \[
      \forall \triangle_1,\triangle_2\in\mathscr{F}(S),\triangle_1\neq\triangle_2\colon\quad \mathscr{E}(Π(\triangle_1)) \cap \mathscr{E}(Π(\triangle_2)) = \emptyset
    \]
    $(S,Π)$ is called an oriented surface mesh.
  \end{definition}

  \begin{definition}[Discrete Surface Mesh Curve]
    Let $n\in\setNatural$, $S$ be a surface mesh, and $x\in \boxBrackets{\bigcup\mathscr{E}(S)}^{n+1}$ be a tuple of $n+1$ points on the edges of the surface mesh $S$, such that adjacent points lie on the same triangle.
    \[
      \forall k\in\setNatural,k\leq n\colon x_k\neq x_{k+1} \land \exists \triangle\in\mathscr{F}(S)\colon x_k, x_{k+1}\in\triangle
    \]
    The (topological) curve given by connecting adjacent points by a straight line is called the (topological) discrete surface mesh curve.
  \end{definition}

  \begin{definition}[Surface Mesh Curve without Artifacts]
    Let $S$ be a polyhedral surface and γ be a surface mesh curve characterized by $x\in\mathscr{M}^{n+1}(S)$.
    \[
      \forall k\in\setNatural, 1\leq k \leq n\colon
      \mathscr{F}_S(x_{k-1}) \cap \mathscr{F}_S(x_{k+1}) = \emptyset
    \]
    If γ is a closed curve, that is $x_0 = x_{n+1}$, then
    \[
      \mathscr{F}_S(x_1) \cap \mathscr{F}_S(x_n) = \emptyset
    \]
  \end{definition}

  \begin{definition}[Total Vertex Angle]
    \[
      ϑ_S(v) \define \sum_{\triangle\in\mathscr{T}, v\in\triangle} ϑ_\triangle(v)
    \]
  \end{definition}

  \begin{definition}[Curve Angle]
    Let $S$ be a polyhedral surface and $p,x,q\in\mathscr{M}(S)$, $x\in S^\circ$ be adjacent vertices of surface mesh curve without artifacts on $S$.
    Let $\mathscr{V}_S(x)$ the neighboring vertices.
    The neighbor set can be partitioned into three possible empty tuples, such that $p,(x),q$ is surface mesh curvature without artifacts.
    There exists two tuples $(p=x_{λ(0)},x_{λ(1)},\ldots,x_{λ(n)},x_{λ(n+1)} = q)$ and $(p,x_{ρ(1)},\ldots,x_{ρ(m)},q)$ that characterize a surface mesh curve without artifacts.
    \[
      β_λ = \sum_{k=1}^n \sphericalangle(x_{λ(k-1)},x_{λ(k)},x_{λ(k+1)})
      \separate
      β_ρ = \sum_{k=1}^m \sphericalangle(x_{ρ(k-1)},x_{ρ(k)},x_{ρ(k+1)})
    \]
    The (unoriented) curve angle at $x$ is given by the sum
    \[
      β = \min \set{β_λ, β_ρ}{}
    \]
    If there is an orientation $Π$ for $S$ then the (oriented) curve angle is given by the segmentation which points are connected by directed edge paths in their respective triangles such that $x$ is not contained.
  \end{definition}
  For the oriented case, the curve angle can only be defined for curve vertices that do not lie on the boundary of $S$.
  Also the interpretation of the total vertex angle for boundary points is different.
  This makes a generalization of the oriented discrete curvature at boundary points difficult.

  \begin{definition}[Discrete Geodesic Curvature]
    \[
      κ_\mathrm{g}(γ) = π - \frac{2π}{ϑ_S(γ)}β_S(γ)
    \]
  \end{definition}

  \begin{definition}[Discrete Geodesic Curvature for Boundaries]

  \end{definition}

  \begin{definition}[Discrete Geodesic Problems]

  \end{definition}

% subsection polyhedral_surfaces (end)

% Differential Geometry on Polyhedral Surfaces
% \autocite{polthier2006}

% Curvature Estimation on Surfaces
% \autocite{rusinkiewicz2004}

% Generation of Surface Normals
% \autocite{max1999,meyer2001,jin2005}

% section preliminaries (end)
\end{document}
