\documentclass{stdlocal}
\begin{document}
\section{Preliminaries} % (fold)
\label{sec:preliminaries}

To systematically approach the design and implementation of curve smoothing algorithms for surface meshes, basic knowledge in the topic of differential geometry and its generalization to polyhedral surfaces is administrable.
It allows to rigorously formulate the initial and boundary value problems for discrete geodesics which build the foundation of curve smoothing as geodesics can be interpreted as the class of smoothest curves.
In the following, a brief overview of the main concepts is given.

\subsection{Differential Geometry} % (fold)
\label{sub:differential_geometry}

  For the purpose of this thesis, we will provide the classical formulation of differential geometry.

  \autocite{goldhorn2009}
  \autocite{carmo2016}
  \autocite{kuehnel2013}
  \autocite{stahl2013}

  The surfaces, we are looking at, are given by compact orientable Riemannian submanifolds (with boundary) of dimension two embedded in $\setReal^3$.

  \begin{definition}[Parameterized Curves]
    Let $n\in\setNatural$, $k\in\setNatural_{\infty}$, and $[a,b]\subset\setReal$ be a compact interval.
    Then a $k$-times continuously differentiable function $\function{γ}{[a,b]}{\setReal^n}$ is called an $n$-dimensional parameterized curve of class $\mathrm{C}^k$.
    \begin{itemize}
      \item $\text{γ is closed} :\iff γ(a) = γ(b)$
      \item $\text{γ is simple} :\iff \forall t\in[a,b)\colon \forall s\in(t,b],(t,s)\neq(a,b)\colon\quad γ(t)\neq γ(s)$
      \item $\text{γ is regular} :\iff \forall t\in [a,b]\colon\quad γ'(t)\neq 0$
      \item $\text{γ is parameterized by arc length} :\iff \norm{γ'} = 1$
    \end{itemize}
    For brevity, a parameterized curve will be called curve and the curve's dimensionality will be left out if the context is clear about or not requiring those signifiers.
  \end{definition}
  Every curve that is parameterized by arc length is automatically a regular curve.
  A regular parameterized curve is by definition a $\mathrm{C}^1$-embedding into $\setReal^n$.
  As a consequence, the image of a regular curve is a 1-manifold of class $\mathrm{C}^1$ embedded in $\setReal^n$.
  The statement that a curve is simple makes sure that other intersections than the first with the last points are not allowed to occur.
  A closed curve can therefore still be simple.

  \begin{definition}[Length of Curves]
    Let $n\in\setNatural$, $k\in\setNatural_{\infty}$, $[a,b]\subset\setReal$ be a compact interval, and $\function{γ}{[a,b]}{\setReal^n}$ be an $n$-dimensional parameterized curve of class $\mathrm{C}^k$.
    Then the length of γ is defined as follows.
    \[
      L(γ) \define \integral{a}{b}{\norm{γ'(t)}}{t}
    \]
    For convenience, we also define the arc length of γ by the following function.
    \[
      \function{s}{[a,b]}{[0,L(γ)]},\quad s_γ(t) \define \integral{a}{t}{\norm{γ'(x)}}{x}
    \]
  \end{definition}
  The function $\norm{γ'}$ is continuous on $[a,b]$ and, consequently, uniformly continuous.
  This means that $L(γ)$ is well-defined for all parameterized curves.
  According to the fundamental theorem of calculus, $s_γ$ is continuously differentiable with derivative $s_γ' = \norm{γ'}$ and surjective.
  For curves parameterized by arc length, it holds that $L(γ)=b-a$.

  \begin{lemma}[Regular curves can be parameterized by arc length]
    Let $n\in\setNatural$, $k\in\setNatural_{\infty}$, $[a,b]\subset\setReal$ be a compact interval, and $\function{γ}{[a,b]}{\setReal^n}$ be an $n$-dimensional parameterized curve of class $\mathrm{C}^k$ that is regular.
    Then up to a constant shift, there exists a unique $k$-times continuously differentiable and bijective function $\function{u}{[c,d]}{[a,b]}$ on a compact interval $[c,d]\subset\setReal$ with strictly positive derivative, such that the composition $γ\composition u$ is a curve parameterized by arc length.
  \end{lemma}
  \begin{proof}
    The restriction of $\norm{\cdot}$ to $\setReal^n\setminus\set{0}{}$ is an infinitely differentiable function.
    The regularity of γ implies that $\norm{γ'}$ and, as a direct consequence, the derivative of $s_γ$ are strictly positive.
    Hence, $s_γ$ is bijective with $k$-times continuously differentiable inverse.
    The derivative of $s_γ^{-1}$ must also be strictly positive and continuous as a composition of continuous functions.
    \[
      \roundBrackets{s_γ^{-1}}' = \frac{1}{s_γ'\composition s_γ^{-1}} = \frac{1}{\norm{γ'}\composition s_γ^{-1}} > 0
    \]
    Define $\tilde{γ}\define γ\composition s_γ^{-1}$.
    Then by differential calculus, it follows, that $\tilde{γ}$ is $k$-times continuously differentiable and parameterized by arc length.
    \[
      \norm{\tilde{γ}'}
      = \norm{γ'\composition s_γ^{-1} \cdot \roundBrackets{s_γ^{-1}}'}
      = \norm{\frac{γ'\composition s_γ^{-1}}{s_γ'\composition s_γ^{-1}}}
      = \norm{\frac{γ'\composition s_γ^{-1}}{\norm{γ'}\composition s_γ^{-1}}}
      = \frac{\norm{γ'\composition s_γ^{-1}}}{\norm{γ'\composition s_γ^{-1}}}
      = 1
    \]
    This shows that $s_γ^{-1}$ fulfills the required properties and the existence of a parameterization by arc length.
    To show its uniqueness up to a constant shift, assume an arbitrary function $u$ as given in the lemma.
    Applying the same reasoning as for $s_γ$, it is clear that $\inverse{u}$ also is $k$-times continuously differentiable with strictly positive derivative.
    By calculation, we may then show its connection to the arc length.
    \[
      1
      = \norm{(γ\composition u)'}
      = \norm{γ'\composition u} \cdot u'
      = \frac{\norm{γ'\composition u}}{\roundBrackets{\inverse{u}}'\composition u}
      = \frac{\norm{γ'}}{\roundBrackets{\inverse{u}}'}
      \quad\implies\quad
      \norm{γ'} = \roundBrackets{\inverse{u}}'
    \]
    To integrate this formula, let $t\in[a,b]$ be arbitrary and use again the fundamental theorem of calculus.
    \[
      s_γ(t)
      = \integral{a}{t}{\norm{γ'(x)}}{x}
      = \integral{a}{t}{\roundBrackets{\inverse{u}}'(x)}{x}
      = \inverse{u}(t) - \inverse{u}(a)
      = \inverse{u}(t) - c
    \]
    \[
      u = s_γ^{-1}\composition s_γ\composition u
      = s_γ^{-1}(\cdot - c)\composition \inverse{u}\composition u
      = s_γ^{-1}\roundBrackets{\cdot - c}
    \]
    This shows that, up to a constant shift, $u$ is the same as $s_γ^{-1}$ and the uniqueness of the parameterization by arc length.
  \end{proof}

  \begin{definition}[Canonical Parameterization by Arc Length]
    Let γ be a regular curve.
    Then its canonical parameterization by arc length $\bar{γ}$ is defined by the following expression.
    \[
      \bar{γ} \define γ\composition s_γ^{-1}
    \]
  \end{definition}

  This lemma shows that, in the sense of regular curves, it is sufficient to handle curves that are parameterized by arc length.
  Only for regular curves, it is possible to talk about their curvature.

  \begin{definition}[Curvature of Regular Curves]
    Let γ be a curve parameterized by arc length and φ be a regular curve.
    Their respective curvatures $κ(γ)$ and $κ(φ)$ are defined by the following expressions.
    \[
      κ(γ)\define\norm{γ''}
      \separate
      κ(φ)\define κ(\bar{φ})\composition s_φ
    \]
  \end{definition}

  \begin{definition}[Geodesic and Normal Curvature of Curves]
    Let $M$ be a $k$-dimensional Riemannian submanifold embedded in $\setReal^n$ and $\function{γ}{[a,b]}{M}$ a curve parameterized by arc length.
    The geodesic curvature $κ_\mathrm{g}(γ)$ and the normal curvature $κ_\mathrm{n}$ of γ arg given by the following expressions.
    \[
      κ_\mathrm{g}(γ)(t) = \norm{P_{\mathrm{T}_{γ(t)}M} γ''(t)}
      \separate
      κ_\mathrm{n}(γ)(t) = \norm{P_{\mathrm{N}_{γ(t)}M} γ''(t)}
    \]
  \end{definition}

  \begin{corollary}
    Let $M$ be a $k$-dimensional Riemannian submanifold embedded in $\setReal^n$ and $\function{γ}{[a,b]}{M}$ a curve parameterized by arc length.
    \[
      κ^2(γ) = κ_\mathrm{g}^2(γ) + κ_\mathrm{n}^2(γ)
    \]
  \end{corollary}

  \begin{definition}[Oriented Curvatures of Curves]
    Let $M$ be a two-dimensional oriented Riemannian submanifold embedded in $\setReal^3$ and $\function{γ}{[a,b]}{M}$ a curve parameterized by arc length.
    Let $N$ be a field of positively oriented unit normals on $M$ and $T$ be a field of tangent vectors such that for all $t\in[a,b]$, $(γ'(t),T(t))$ is positively oriented basis in $\mathrm{T}_{γ(t)}M$.
    \[
      κ_\mathrm{g}(γ) = \scalarProduct{T}{γ''}
      \separate
      κ_\mathrm{n}(γ) = \scalarProduct{N}{γ''}
    \]
  \end{definition}

  \begin{definition}[Geodesics]
    Let $M$ be a two-dimensional Riemannian submanifold embedded in $\setReal^3$ and $\function{γ}{[a,b]}{M}$ a curve parameterized by arc length.
    The curve γ is called a geodesic if $κ_g(γ) = 0$.
  \end{definition}

  \begin{definition}[Geodesics Problems]
    Let $M$ be a two-dimensional Riemannian submanifold embedded in $\setReal^3$.

    The problem of finding a geodesic $\function{γ}{[a,b]}{M}$ for a given starting point $γ(a)$ and a given direction $γ'(a)$ is called the geodesic initial value problem.

    The problem of finding a geodesic $\function{γ}{[a,b]}{M}$ for given start and end points $γ(a)$ and $γ(b)$ is called the geodesic boundary value problem.
  \end{definition}

% subsection differential_geometry (end)

\subsection{Polyhedral Surfaces} % (fold)
\label{sub:polyhedral_surfaces}

  There are definitions for manifolds with corners.
  In the topological case, these are equivalent to manifolds with boundaries.
  For the smooth case, it is different.
  The half space is only homeomorphic and not diffeomorphic to the cube space.
  so, we could generalize a polyhedral surface as a smooth manifold with corners.
  on the definitions, there is no uniform agree.
  So, we will not use the generalization of manifolds with corners.
  we will use the theory of polthier which should be applicable to manifolds with corners.

  \begin{definition}[Triangle]
    Let $n\in\setNatural$ with $n\geq 2$ and $A,B,C\in\setReal^n$ affinely independent points.
    Then the (topological) triangle $\triangle$ (embedded in $\setReal^n$) is given by the following expression.
    \[
      \triangle \define \set{uA + vB + wC}{u,v,w\in[0,1], u+v+w=1}
    \]
    The triangle's vertices $\mathcal{V}(\triangle)$ and edges $\mathcal{E}(\triangle)$ are hereby defined as follows.
    \[
      \mathcal{V}(\triangle) \define \set{A,B,C}{}
      \separate
      \mathcal{E}(\triangle) \define \set{\overline{AB}, \overline{BC}, \overline{CA}}{}
    \]
    \[
      ϑ_\triangle(A) = \arccos\frac{\scalarProduct{\overrightarrow{AB}}{\overrightarrow{AC}}}{\norm{\overrightarrow{AB}}\norm{\overrightarrow{AC}}}
    \]
  \end{definition}
  A triangle is topological 2-manifold with boundary embedded in $\setReal^n$.
  Triangles are planar surfaces and, hence, geodesics are given by straight lines.

  \begin{definition}[Surface Mesh]
    Let $n\in\setNatural$ with $n\geq 2$ and $\mathcal{T}\neq\emptyset$ be a finite set of triangles embedded in $\setReal^n$.
    Let $S=\cup\mathcal{T}$ be a two-dimensional topological manifold (with boundary), such that for all $\triangle_1,\triangle_2\in\mathcal{T}$ with $\triangle_1\neq\triangle_2$ the following holds.
    \[
      \triangle_1\cap\triangle_2 \in \set{\emptyset}{} \cup [\mathcal{V}(\triangle_1)\cap\mathcal{V}(\triangle_2)] \cup [\mathcal{E}(\triangle_1)\cap\mathcal{E}(\triangle_2)]
    \]
    In this case, $S$ is called a (topological) surface mesh (embedded in $\setReal^n$).
    With $\mathcal{V}(S)$ we denote its vertices, with $\mathcal{E}(S)$ its edges, and with $\mathcal{F}(S)$ its faces.
    \[
      \mathcal{V}(S) \define \bigcup_{\triangle\in\mathcal{T}} \mathcal{V}(\triangle)
      \separate
      \mathcal{E}(S) \define \bigcup_{\triangle\in\mathcal{T}} \mathcal{E}(\triangle)
      \separate
      \mathcal{F}(S) \define \mathcal{T}
    \]
  \end{definition}
  A surface mesh is again a topological 2-manifold.

  \begin{definition}[Piecewise Geodesic Curves]

  \end{definition}

  \begin{definition}[Total Vertex Angle]
    \[
      ϑ_S(v) \define \sum_{\triangle\in\mathcal{T}, v\in\triangle} ϑ_\triangle(v)
    \]
  \end{definition}

  \begin{definition}[Discrete Geodesic Curvature]
    \[
      κ_\mathrm{g}(γ) = π - \frac{2π}{ϑ_S(γ)}β_S(γ)
    \]
  \end{definition}

  \begin{definition}[Discrete Geodesic Problems]

  \end{definition}

% subsection polyhedral_surfaces (end)

% Differential Geometry on Polyhedral Surfaces
% \autocite{polthier2006}

% Curvature Estimation on Surfaces
% \autocite{rusinkiewicz2004}

% Generation of Surface Normals
% \autocite{max1999,meyer2001,jin2005}

% section preliminaries (end)
\end{document}
