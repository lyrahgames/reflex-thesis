\documentclass[crop=false]{stdlocal}
\begin{document}
\section{Conclusions, Limitations, and Future Work} % (fold)
\label{sec:conclusions}

  During the course of this thesis, I build on the theory of \textcite{polthier2006} and formulated a rigorous mathematical foundation for the handling of discrete geodesics and curve smoothing algorithms on polyhedral surfaces.
  Hereby, special emphasis has been put on the consideration of boundaries which are are often neglected by the general literature.
  Even on smooth manifolds, in the presence of boundaries the concept of locally shortest geodesics no longer agrees with the definition of straightest geodesics.
  This problem cannot be eliminated by simply redefining geodesic curvature on boundaries to be zero as this would lead to unintuitive results.
  State-of-the-art literature concerned with algorithms to generate geodesic paths and distances considers the concepts for discrete locally shortest and straightest geodesics as the de facto standard for the generalization of geodesics onto polyhedral surfaces.
  Regarding future work, I also want to consider other formulations of geodesics which may lead to a more intuitive interpretation and the removal of inconsistencies.

  % Following the mathematical foundation, different metrics for measuring the smoothness and similarity of curves have been proposed to allow for objective and quantitative comparisons of different curve smoothing states and techniques.
  % The largest problem is the uniform weighting along the length of a curve.
  % A curve moving with respect to its curvature gets much shorter at parts with a higher curvature value.
  % A general length mapping would be more intuitive.

  In the practical part of this work, I successfully designed and implemented various data structures to represent surface mesh curves as well as algorithmical primitives for their general movement and smoothing along polyhedral surfaces.
  All the implementation-specific details have been given as C++ code snippets to allow for reproducibility.
  The algorithmical primitives for smoothing surface mesh curves build on the work of \textcite{martinez2005} and \textcite{lawonn2014} by using data structures from \textcite{mancinelli2022} and a consistent curvature propagation.
  Additionally, the calculation of desired geodesic curvature values has been improved by using a weighted stencil of adjacent geodesic curvature values and applying it multiple times to geodesic curvatures of the initially given curve.
  % The main smoothing algorithm is based on an iterative approach to locally optimize geodesic curvature values in conjunction with surface-dependent penalty values to keep the outcome close to its original.
  The main smoothing algorithm is based on an iterative approach to locally optimize these geodesic curvature values.
  As such, it is part of the variational domain of solutions to curve smoothing.
  % This also allows for more general considerations such as anisotropic surface metrics.
  % Together with the defined smoothing metrics, I was able to proof that the algorithm indeed smoothes an initial curve and that it converges.

  % I have stated the concrete data structures for polyhedral surfaces and surface mesh curves and also added primitive operations.
  % This allowed for general construction on how to move a surface mesh curve along a polyhedral surface.
  % Also, I propose a more robust method for generating desired geodesic curvature values by using a weighting stencil.
  % Instead of an inconsistent curvature propagation, I have introduce curvature mapping.
  % The largest limitation of consistently moving surface mesh curves along polyhedral surface is the lack of mathematical convergence properties concerning discrete geodesic curvature.

  % The algorithms have been implemented as parallelized version on the CPU and GPU with the standard thread library of the C++ STL and the compute shaders of OpenGL, respectively.
  % I have designed tests based on the \citetitle{thingi10k} dataset to evaluate their robustness and efficiency.
  % Even for polyhedral surfaces consisting of millions of triangles, surface mesh curves exhibit typically only exhibit thousands of control points.
  % Most of curve smoothing or straightening algorithms are not parallelizable.
  % Due to iterative nature and the convergence behavior, I was able to show that CPU-based and GPU-based parallelization indeed allows for faster determination of smoothed curves.
  % Still, the smoothing of curves seems not to be the bottleneck of the application.
  % So, a parallelization for only a few thousands points might quickly become infeasible.
  % As there are data and flow dependencies in a curve smoothing procedure, parallelization is limited as it needs to constantly communicate with the CPU.
  % A bigger emphasis should lie on the robustness.

  As the movement of surface mesh curves depends on the topology of the polyhedral surface, the movement speed and convergence of algorithms can be slow for high-resolution surface meshes.
  Each iteration, a surface mesh curve may only propagate one vertex ahead.
  Especially for geodesics tracing, where the accumulated amount a curve may change is maximal, this process might be inefficient.
  Furthermore, the algorithmical primitives and smoothing procedure cannot considered to be robust if the polyhedral surface exhibits faces or vertices that are numerically extreme.
  Other methods, based on freely moving a curve in space followed by a projection onto the surface, could exhibit a better convergence rate.

  Even for high-resolution polyhedral surfaces consisting of millions of triangles, surface mesh curves typically exhibit a few thousand control points.
  As such, the process of smoothing surface mesh curves does not appear to be the bottleneck of the overall program pipeline.
  Besides, most of the provided primitives for algorithms exhibit more complicated data and flow dependencies which hinder a parallelization on the CPU and GPU.
  As a result, the parallelization of the smoothing algorithm is expected to be unsuitable for general applications due to the high complexity of programming involved in comparison to the benefits of efficiency gain.
  Instead, future work should put serious focus on a robust implementation that would be able to handle many artificial and degenerate case of polyhedral surfaces that often arise as smaller artifacts in real-world surface meshes.
  Such an implementation should be thoroughly tested by constructing generic tests that can be applied to a whole set of polyhedral surfaces such as the \citetitle{thingi10k} dataset.
  % Apart from that, regarding the iterative nature of the main smoothing algorithm, a parallelization on the CPU could be realized by using a pipeline which

  Another problem that arises when dealing with general smoothing algorithms for surface mesh curves is the fact that smoothing a curve is not a clear goal.
  There is no universally agreed-upon metric to measure the quality of a smoothing process.
  Implementing stopping criteria for algorithms or assigning desired geodesic curvature values are subjective processes and only provide heuristics to make an initially given curve intuitively smoother.
  However, for a rigorous design of smoothing, future work should involve the research on the generalization of norms that complete the spaces of $\mathrm{C^k}$-curves.
  Not only taking the maximum, average, mean-squared geodesic curvature into account but also the values of its first or second derivative could allow to construct improved algorithms whose smoothing capabilities are provable.
  Also the proof of convergence for alternative smoothing algorithms, as it was done by \textcite{lawonn2014}, should be part of future work.

  Expansion of curves, not only shortening

  Alternative Smoothing Algorithms: Your algorithm to lift the polyhedral surface into four dimensions by adding a scalar potential for geodesics.

  Anisotropic Metrics

  Application to medicine

% section conclusions (end)
\end{document}
