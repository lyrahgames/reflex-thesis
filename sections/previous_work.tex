\documentclass{stdlocal}
\begin{document}
\section{Previous Work} % (fold)
\label{sec:previous_work}

  As marked in the introduction in section~\ref{sec:introduction}, the smoothing of curves on surface meshes is an essential operation for mesh processing and, as a consequence, for many other domain areas, like computer graphics, image-based medicine, and engineering, that rely on such tools \autocite{ji2006,kaplansky2009}.
  In most of its applications, initial curves are either provided by means of direct user interaction or by automatic or semiautomatic feature detection algorithms \autocite{zachow2003,lawonn2014}.
  The finite precision of the underlying surface mesh together with all the steps included to define an initial curve usually makes resulting lines contain non-smooth artifacts which may violate given constraints or expected properties and therefore degrade its quality \autocite{kaplansky2009,lawonn2014}.
  Introducing a smoothing stage into the curve processing pipeline, the mesh segmentation is expected to be of much higher quality which greatly increases its usage for areas like machine learning \autocite{benhabiles2011} or medicine \autocite{zachow2003,alirr2019}.
  During the last two decades, there have been multiple successful attempts for constructing algorithms to smooth curves on surfaces \autocite{hofer2004,bischoff2005,lawonn2014,mancinelli2022}.
  In this section, a brief overview of their major contributions is given.

  As stated in the previous section~\ref{sec:preliminaries}, a crucial tool for working with curves on two-dimensional manifolds is the ability to generate geodesics in the sense of the initial and boundary value problem.
  The rigorous mathematical concepts and definitions for the discrete geodesics problems have been elaborated by \textcite{mitchell1987} and \textcite{polthier2006} first published in 1997.
  Additionally, \textcite{mitchell1987} built an algorithm to solve the discrete boundary value problem, that used a continuous version of the algorithm of \textcite{dijkstra1959} to find the shortest path connecting two given points.
  Furthermore, \textcite{polthier2006} provided an iterative algorithm to solve the discrete initial value problem of finding the geodesic given a starting point and a direction.
  They also introduced the parallel translation of vectors along the surface for particle transportation.
  This algorithm has been improved by \textcite{mancinelli2022} through the use of optimized data structures and a superior choice of initial curves.
  Based upon the theory of \textcite{polthier2006}, \textcite{martinez2005} provided an algorithm to the discrete boundary value problem.
  Hereby, a starting curve on the surface had to be given as initial value to iteratively improve it up to an approximated geodesic.
  \textcite{surazhsky2005} developed exact and approximate algorithms based on \textcite{mitchell1987} for the discrete initial and boundary value problem, which could be evaluated efficiently by the use of distance fields.
  Extending the idea of distance fields as an intermediate step to the generation of geodesics, \textcite{bommes2007} generalized the algorithm of \textcite{surazhsky2005} to not only handle isolated points for their distance fields but also general polygons on the surface.
  Also based on the results of \textcite{mitchell1987}, \textcite{kimmel1996} introduced the so-called fast marching approach, which used the eikonal equation to build propagating fronts to more efficiently generate the distance fields.
  Hereupon, \textcite{crane2013} also used the gradient of the heat kernel to reconstruct a distance field by solving the Poisson equation.


  Application of Cutting Curves in Mesh Processing
  \autocite{zachow2003}
  \autocite{benhabiles2011}
  \autocite{ji2006}

  Previous Intuitive Approach called Corner Cutting in Planar
  \autocite{chaikin1974}
  \autocite{dyn1992}
  in space
  \autocite{morera2008}
  But this may not provide a real surface curve.

  Lines as Snakes
  \autocite{kass1988}
  Generalization 2D-Manifold
  \autocite{bischoff2005}
  \autocite{jung2004}

  Automatic Surface Segmentation and Cutting
  \autocite{lee2002}
  \autocite{lee2004}

  Feature-Sensitive Curve Smoothing
  \autocite{lai2007}

  Geodesics on Triangular Meshes
  \autocite{martinez2005}

  Bezier Curves on Meshes
  \autocite{martinez2007}
  and splines in manifolds
  \autocite{hofer2004}
  and geodesics in polyhedral surfaces
  \autocite{polthier2006}
  \autocite{mitchell1987}
  \autocite{surazhsky2005}

  geodesic distance fields
  \autocite{bommes2007}
  \autocite{kimmel1996}

  heat maps
  \autocite{crane2013}

  \autocite{dijkstra1959}

  \autocite{ma2007}
  \autocite{pottmann2005}
  \autocite{levy2002}

  \autocite{mancinelli2022}

  \autocite{yu2021}

  \autocite{engelke2018}

% section previous_work (end)
\end{document}
