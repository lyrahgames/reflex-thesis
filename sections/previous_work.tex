\documentclass{stdlocal}
\begin{document}
\section{Previous Work} % (fold)
\label{sec:previous_work}

As marked in the introduction in section~\ref{sec:introduction}, the smoothing of curves on surface meshes is an essential operation for mesh processing and, as a consequence, for many other domain areas, like computer graphics, image-based medicine, and engineering, that rely on such tools \autocite{ji2006,kaplansky2009}.
In most of its applications, initial curves are either provided by means of direct user interaction or by automatic or semiautomatic feature detection algorithms \autocite{zachow2003,lawonn2014}.
The finite precision of the underlying surface mesh together with all the steps included to define an initial curve usually makes resulting lines contain non-smooth artifacts which may violate given constraints or expected properties and therefore degrade its quality \autocite{kaplansky2009,lawonn2014}.
Introducing a smoothing stage into the curve processing pipeline, the mesh segmentation is expected to be of much higher quality which greatly increases its usage for areas like machine learning \autocite{benhabiles2011} or medicine \autocite{zachow2003,alirr2019}.
During the last two decades, there have been multiple successful attempts for constructing algorithms to smooth curves on surfaces \autocite{hofer2004,bischoff2005,lawonn2014,mancinelli2022}.
In this section, a brief overview of their major contributions is given.

As stated in the previous section~\ref{sec:preliminaries}, a crucial tool for working with curves on two-dimensional manifolds is the ability to generate geodesics in the sense of the initial and boundary value problem.
The rigorous mathematical concepts and definitions for the discrete geodesics problems have been elaborated by \textcite{mitchell1987} and \textcite{polthier2006} first published in 1997.
Additionally, \textcite{mitchell1987} built an algorithm to solve the discrete boundary value problem, that used a continuous version of the algorithm of \textcite{dijkstra1959} to find the shortest path connecting two given points.
Furthermore, \textcite{polthier2006} provided an iterative algorithm to solve the discrete initial value problem of finding the geodesic given a starting point and a direction.
They also introduced the parallel translation of vectors along the surface for particle transportation.
This algorithm has been improved by \textcite{mancinelli2022} through the use of optimized data structures and a superior choice of initial curves.
Based upon the theory of \textcite{polthier2006}, \textcite{martinez2005} provided an algorithm to the discrete boundary value problem.
Hereby, a starting curve on the surface had to be given as initial value to iteratively improve it up to an approximated geodesic.
\textcite{surazhsky2005} developed exact and approximate algorithms based on \textcite{mitchell1987} for the discrete initial and boundary value problem, which could be evaluated efficiently by the use of distance fields.
Extending the idea of distance fields as an intermediate step to the generation of geodesics, \textcite{bommes2007} generalized the algorithm of \textcite{surazhsky2005} to not only handle isolated points for their distance fields but also general polygons on the surface.
Also based on the results of \textcite{mitchell1987}, \textcite{kimmel1996} introduced the so-called fast marching approach, which used the eikonal equation to build propagating fronts to more efficiently generate the distance fields.
Hereupon, \textcite{crane2013} also used the gradient of the heat kernel to reconstruct a distance field by solving the Poisson equation.

For the actual creation of smooth curves on surfaces, evidence shows that only a few main approaches have emerged.
Presumably, the most intuitive way for a curve smoothing algorithm to work is by using subdivision schemes for polygonal lines, also called corner cutting.
The algorithm thereby subdivides each line segment and positions newly created points in such a way that the resulting curve is smoother than the previous one.
First introduced and used in the planar case by \textcite{chaikin1974} and \textcite{dyn1992}, \textcite{morera2008} generalized the algorithm to polygonal lines on surfaces.
Unfortunately, the subdivision curve may consist of points located anywhere inside the faces of the underlying mesh.
Hence, its trajectory is not a surface curve in the strong rigorous mathematical sense and might miss essential parts of the mesh.
The objective to construct a robust curve smoothing algorithm dictates that for discrete surfaces all line segments should lie inside a face of the surface.

The smoothing of curves based on features of the surface mesh for automatic mesh segmentation and cutting has been shown to successfully work by \textcite{jung2004}, \textcite{bischoff2005}, and \textcite{lai2007}.
To classify surface features, \textcite{lai2007} used a feature-sensitive curve smoothing which allowed them to obtain smooth boundaries for mesh features.
Both publications, \textcite{jung2004} and \textcite{bischoff2005}, are building upon the previous work of \textcite{lee2002} and \textcite{lee2004}.
They generalized so-called snakes for two-dimensional manifolds to represent curves on surfaces that are able to find crucial mesh features by providing an initial curve.
First introduced by \textcite{kass1988} for two-dimensional images, snakes are closed curves that evolve over many iterations to the features of the mesh by minimizing internal and external forces based on curvature, length, and distance to features.
For snakes, the initial shape is completely unimportant.
They are allowed to merge or split, such that a rapid movement towards the features of the mesh is to be expected.
As a direct consequence, feature-based curve smoothing does not allow for arbitrary trajectories and would lead to a curve that may not be assumed to be near the original curve, which makes these algorithms bad candidates for general curve smoothing.

Another approach for representing and generating smooth curves on surfaces is through the use of splines.
\textcite{hofer2004} and \textcite{pottmann2005} determined splines in general manifolds by addressing the design of curves as an optimization problem in the sense of minimizing the curve's overall quadratic energy and using a variational approach to compute a solution.
Their approach is not only applicable to curves on surface meshes but can be used for a much broader variety of cases, including for example the design of rigid body motions.
Alas, for a small number of control points, the resulting curve may still not be assumed to exhibit a close distance to the initially selected curve.
Overcoming this issue would involve adding many more control points and, eventually, a much higher burden for the user who would need to define those points.
According to \textcite{mancinelli2022}, the variational approach to solve the optimization problem for the design of curves is also expected to provide a poor performance for surface meshes that consist of millions of triangles.

These results quickly lead to the representation of smooth curves by using generalized Bézier splines.
The main contributions are given by \textcite{martinez2007} and \textcite{mancinelli2022}.
They lift the concept of Bézier curves in the two-dimensional Euclidean space to geodesic Bézier splines located in the surface.
The initially chosen curve samples are thereby used as control points to determine the individual shapes of the Bézier splines.
\textcite{mancinelli2022} successfully showed their approach to be superior to other spline alternatives and provided real-time performance when tracing the trajectories of the given splines on the surface for even high-resolution meshes.
In addition, they offer a robust implementation of their algorithm in an open-source C++ framework, named \citetitle{yoctogl} \autocite{agus2019}, that can be found on GitHub \footfullcite{yoctogl}.
Nevertheless, their approach exhibits similar issues compared to the variational spline approach in that only the use of many control points will make sure that the resulting curve will be near to the initially defined curve.
In this sense, the approach of \textcite{mancinelli2022} does not fit our need by being specialized for vector graphics on surface meshes.

Generalizing on the iterative algorithm of \textcite{martinez2005}, \textcite{lawonn2014} created an algorithm for curve smoothing based on curvature values given for each trajectory point.
Each iteration, the algorithm tries to locally fulfill the given curvature constraint, eventually converging to its final smooth curve.
The algorithm guarantees a close distance to the initial curve and can also be used with a simplified user interaction where only one parameter has to be adjusted.
During the process, all iterations adapt to the resolution of the underlying surface mesh and directly provide the curve on the surface without the need of tracing or projection.
The algorithm was shown to be robust against geometric and parametric noise and applied in a medical context, where domain experts evaluated its usability.
\textcite{lawonn2014} proved the convergence of the algorithm and also compared the quality of the generated smoothed curves against the spline-based variational approach without any issue.
Furthermore, no surface normals or curvature, that would need to be evaluated first, is needed for the algorithm.
As a result, it may also be formulated for generalized two-dimensional triangular manifolds leaving a wide variety of mesh data structures to choose from \autocite{guibas1985}.
Unfortunately, \textcite{lawonn2014} do not provide any language-specific implementation or performance evaluation.

According to the explanations and descriptions above, for the purpose of this thesis, our design and implementation will mainly focus on the approach given by \textcite{lawonn2014}.
Their algorithm seems to fit our needs in nearly all important aspects.
To solve the potential performance issue and get real-time behavior, a parallelization on the CPU and GPU will be carried out.
We will also strive for an optimized curve initialization and geodesics generation that builds upon the basic building blocks of the approach given by \textcite{mancinelli2022}.

% \autocite{ma2007}
% \autocite{levy2002}
% \autocite{yu2021}
% \autocite{engelke2018}

% section previous_work (end)
\end{document}
