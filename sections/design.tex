\documentclass{stdlocal}
\begin{document}
\section{Design} % (fold)
\label{sec:design}

\subsection{Overview and Program Pipeline} % (fold)
\label{sub:program_pipeline}
  The implementation will not be given for all parts of the program pipeline.
  These will still be briefly discussed in this section.
  \begin{itemize}
    \item STL Input (get the topological connections)
    \item Mesh Preprocessing for Normals and Neighbors and Checks for Orientability and Connection $\rightarrow$ Render the mesh
    \item Initial Curve Selection (Vertex Curve, Face Curve, Tracing of Geodesics) $\rightarrow$ Render the initial curve
    \item Parameter Set Selection for Smoothing Algorithm (Distance Envelope, Mask of Allowed Vertices, Tolerance, Iterations, Smoothness, Epsilon Hull Smoothing) $\rightarrow$ Render the envelopes and estimations based on parameters
    \item Curve Smoothing and Smooth Curve Properties $\rightarrow$ Render smoothed curve
    \item Smoothed Curve Transformation for Cutting, Storing, and other Applications
  \end{itemize}
% subsection program_pipeline (end)

\subsection{Mesh Preprocessing} % (fold)
\label{sub:mesh_preprocessing}
  We need the topological connections of a given surface mesh.
  In the case of general file formats, a scene is typically separated into multiple meshes which are topologically but not smoothly connected.
  For proper drawing of curves in a whole scene, the topological structure of connections needs to be generated first.
  Using the STL file format, this is not needed.
  A given Mesh for the algorithm can be quite general.
  Providing triangles which will introduce numerical difficulties is out of the scope of this thesis.
  We will focus on orientable meshes.
  Please note, that is not an actual restriction for the algorithm but will only speed up the implementation.
  Furthermore, we look at surfaces which are the boundary of volumes in three-dimensional Euclidean space which can be looked at as open submanifolds.
  These volumes are therefore by definition oriented and, hence, their boundary needs to be, too.
  Still, we need to look boundaries for surfaces as, for example, distance envelopes restrict the surface without boundary to be one.
  The boundaries need to be handled properly.
  Also, these requirements are only need to hold for the restriction to the distance envelope.
  Often, lines will be drawn on oriented/orientable parts of the surface even if the surface is not orientable due to artifacts originating from scanning or generating meshes.
% subsection mesh_preprocessing (end)

\subsection{Initial Curve Selection} % (fold)
\label{sub:initial_curve_selection}
  As for the vast majority of optimization algorithms, the results and efficient working are highly dependent on the chosen starting values.
  Curve smoothing itself is an optimization process and to solve the boundary value problem, we need an initial value.
  So, choosing the initial curve in the right way will also heavily change the speed and quality of the algorithm.
  The algorithm should be able to handle a vast amount unsmooth curves.
  Still, the handling of artifacts will be taken care of at the start.
  \paragraph{User Interface for Selecting and Controlling Initial Curves}
  \paragraph{Drawing by Ray Tracing}
  \paragraph{Connecting the Vertices}
  \paragraph{Closed Initial Curves and Fixed Vertices}
  \paragraph{Artifact Removal}
  \paragraph{Smoothed Curvature Values by Stencil}
    Stencil is discrete approximation of solution to Laplacian equation or heat equation which is smooth (infinitely differentiable).
    Maybe it would be useful to keep the length of the curve and still smooth it.
  \paragraph{Vertex Curves}
  \paragraph{Face Curves}
  \paragraph{Tracing Geodesics}
% subsection initial_curve_selection (end)

\subsection{Curve Smoothing Algorithm} % (fold)
\label{sub:curve_smoothing_algorithm}
  \paragraph{Idea and Overview}
  \paragraph{Edge Vertex Relaxation}
    We need to take a look at topological and numerical robustness.
  \paragraph{Vertex Vertex Relaxation}
  \paragraph{Critical Vertex Handling}
  \paragraph{Artifact Removal and Self-Intersection Handling}
  \paragraph{Desired Curvature Mapping}
    Desired curvatures need to be constant along edges.
    Therefore interpolate on angle-basis around vertex.
    We do not want to loose information of desired curvatures.
  \paragraph{Curve Evaluation}
  \paragraph{Correctness and Convergence}
    Correctness can be shown by showing the convergence to the curvature values.
    This by definition of the given curvature values smooths the curve.
    The convergence might only be shown for contracting the curve by smoothing.
    As the limit is the geodesic, the prove of its convergence is there.
  \paragraph{Complexity}
% subsection curve_smoothing_algorithm (end)

% section design (end)
\end{document}
