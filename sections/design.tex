\documentclass{stdlocal}
\begin{document}
\section{Design} % (fold)
\label{sec:design}

Experimenting with various designs and implementations of algorithms, that robustly and efficiently smooth discrete curves on surface meshes, requires us to use a whole program pipeline or framework for loading input data, intuitively working with user interaction, and visualizing any intermediate results.
As changing specific parts of such a pipeline may affect the performance, outcomes, and overall behavior of an algorithm or program, a brief overview over all of its components is essential to allow for reproducible results and will therefore be given in the following subsections.
This will also make it possible for readers to simply reconstruct, adjust, and improve the pipeline for their own domain-specific projects.
Afterwards, I will thoroughly elaborate on the design and implementation of chosen data structures and algorithms together with their mathematical primitives for curve smoothing.

The pipeline or framework, described in this thesis, has been manually implemented with the C++ programming language and the OpenGL graphics API using some external libraries to handle low-level tasks.
Already described in the introduction in section~\ref{sec:introduction}, this choice is well suited for open-source graphics applications and provides programmers with a large freedom when it comes to the implementation of data structures and algorithms.
The complete source code of the project, called \textit{nanoreflex}, is provided as an open-source repository on GitHub.

\subsection{Overview of the Program Pipeline} % (fold)
\label{sub:program_pipeline}
  In the following, the stages of the implemented program pipeline are stated and summarized.
  These stages naturally arise from an insight into the generation and tracing of geodesics on surface meshes, described in section~\ref{sec:previous_work}.
  Locally shortest or straightest geodesics that follow as a solution from the discrete boundary value problem are in some sense the smoothest curves that cannot be made any smoother.
  As we typically need to provide an initial curve for those geodesics to be found, the whole generation and tracing of geodesics can be looked at as an extreme smoothing process.
  Adding parameters to let the user decide on when to stop this process then results in the pipeline given below and schematically shown in figure~\ref{fig:program-pipeline}.

  \begin{figure}[h]
    \begin{center}
      \large
      INSERT YOUR IMAGE HERE!
    \end{center}
    \caption[Program Pipeline Stages]{%
      \textbf{Program Pipeline Stages}\\
      The scheme shows all the stages of the implemented program pipeline.
      The arrows indicate data flow and dependencies.
    }
    \label{fig:program-pipeline}
  \end{figure}

  \begin{enumerate}
    \item \textbf{Surface Mesh Loading:}\\
      Load a surface mesh given by a specific file format from the storage.
    \item \textbf{Surface Mesh Preprocessing:}
      \begin{enumerate}
        \item Generate a connected mesh object to properly model the topology of the input.
        \item Generate the pseudo-normals for each vertex.
        \item Generate the dual graph for the neighbors of each triangle.
        \item Check that the mesh is a valid orientable two-dimensional topological manifold.
      \end{enumerate}
    \item \textbf{Initial Curve Selection:} \\
      Let the user choose an initial curve by drawing on the surface.
    \item \textbf{Parameter Selection:} \\
      Let the user choose parameters for the curve smoothing algorithm.
    \item \textbf{Curve Smoothing:} \\
      Smooth the initial curve according to the constraints given by the selected parameters.
    \item \textbf{Postprocessing:} \\
      Optionally apply the smoothed curve in a domain-specific context.
  \end{enumerate}
  According to figure~\ref{fig:program-pipeline}, the output of every stage is not only forwarded to the next stage but also specifically visualized and rendered to the screen to provide the user with visual feedback and to catch errors or exceptional behavior as early as possible.
  Besides, additional user interactions will be processed for all stages to allow for measurements and adjustments and to be able to react to the output of previous stages.
  The loading of surface meshes from different file formats is typically handled by loader libraries, such as \textit{Assimp}, and will not be further explained.
  On the other hand, surface mesh preprocessing and the initial curve selection are crucial steps in the whole pipeline which remove geometric degeneracies and artifacts that would otherwise violate assertions needed for the correct execution of the algorithm.
  Therefore both stages will be discussed in more detail in the following subsections.
  The postprocessing in our case is optional and only provided for the sake of completeness.
  The parameter selection is highly dependent on the implementation of the curve smoothing algorithm and, as a result, it will be described together with the curve smoothing itself.
% subsection program_pipeline (end)

\subsection{Polyhedral Surface Data Structure} % (fold)
\label{sub:polyhedral_surface_data_structure}
  For loading and preprocessing a surface mesh, the data structure of a polyhedral surface has to be defined first.
  The easiest representation, especially useful for rendering, is to have two separate lists of vertices and faces.
  Each vertex should at least store its position but will most likely also provide a pseudo-normal for better visualization.
  Every face is a triangle, that refers to the three vertices it contains.
  This is the minimal information that needs to be provided to create a polyhedral surface.

  To allow points to move on the surface, each face must know about its neighbors.
  Every triangle can at most have three neighbors.
  So, each triangle will also refer to its neighbors.
  When there is no neighbor, we will store an invalid id.

  This makes the dual graph of the surface mesh available.

  \inputCodeBlock[title = Polyhedral Surface]{code/polyhedral_surface/polyhedral_surface.hpp}

  An optional step would also to construct the vertex graph.
  But we probably do not need it.

  The triangle-based data structured described so far was also used by \textcite{mancinelli2022} and seems to be superior to other edge-based data structures.
  Looking at Delaunay triangulations this was similar.
  So, the quad-edge structure may not be a good idea.

  \begin{figure}[h]
    \begin{center}
      \large
      INSERT YOUR IMAGE HERE!
    \end{center}
    \caption[Polyhedral Surface Data Structure]{%
      \textbf{Polyhedral Surface Data Structure}
      Scheme and Memory Layout of triangles and vertices
    }
  \end{figure}

  \begin{figure}
    \begin{center}
      \large
      INSERT YOUR IMAGE HERE!
    \end{center}
    \caption[Examples of Polyhedral Surfaces]{%
      \textbf{Examples of Polyhedral Surfaces}
      The images shows oriented polyhedral surfaces with and without boundaries, also called models.
    }
  \end{figure}
% subsection polyhedral_surface_data_structure (end)

\subsection{Surface Mesh Preprocessing} % (fold)
\label{sub:mesh_preprocessing}
  The goal of preprocessing a surface mesh which has been loaded from disk is to transform its data into a valid and efficient representation of a polyhedral surface that should exhibit functionality to draw and move points and discrete curves on its surface.
  In the case that this transformation is not possible, the stage at least must check for the validity of the given data and emit an error if there might be any violations.
  % On one hand, the preprocessing of a given surface mesh needs to make sure that the program or algorithm is actually dealing with a valid two-dimensional topological manifold.

  A given Mesh for the algorithm can be quite general.
  Providing triangles which will introduce numerical difficulties is out of the scope of this thesis.

  We will focus on orientable meshes.
  Please note, that is not an actual restriction for the algorithm but will only speed up the implementation.
  Furthermore, we look at surfaces which are the boundary of volumes in three-dimensional Euclidean space which can be looked at as open submanifolds.
  These volumes are therefore by definition oriented and, hence, their boundary needs to be, too.
  Still, we need to look boundaries for surfaces as, for example, distance envelopes restrict the surface without boundary to be one.
  The boundaries need to be handled properly.
  Also, these requirements are only need to hold for the restriction to the distance envelope.
  Often, lines will be drawn on oriented/orientable parts of the surface even if the surface is not orientable due to artifacts originating from scanning or generating meshes.

  \paragraph{Topological Connections}
  We need the topological connections of a given surface mesh.
  In the case of general file formats, a scene is typically separated into multiple meshes which are topologically but not smoothly connected.
  For proper drawing of curves in a whole scene, the topological structure of connections needs to be generated first.
  Using the STL file format, this is not needed.

  \paragraph{Generating the dual graph}
   To generate such a dual graph we refer to this book.
  It is assumed to be a solved problem.
  First, we have created a hash map of oriented edges with a custom hash function to mangle vertex indices.
  The mapped values store the faces indices and their location in the triangle.
  By inverting the indices of a stored edge, one can easily access the neighboring triangle if it exists.
  Furthermore, it is easy to check whether the mesh is oriented and a valid two-dimensional manifold.

  \paragraph{Check for orientability and validity}

% subsection mesh_preprocessing (end)

\subsection{Discrete Surface Curve Data Structure} % (fold)
\label{sub:discrete_surface_curve_data_structure}
  There two main attempts for the construction of a data structure representing surface mesh curves.
  One of them is based on edges and the other on triangles.

  We do not look at polyhedral surfaces as an approximation of a real-world smooth manifold.

  Advantages of discrete surface mesh curves over more general curves on polyhedral surfaces
  \begin{itemize}
    \item Curve is provided in triangle precision. Better Compression. More points would be waste of memory.
    \item Robust: Snap to vertex if points get arbitrary near to each other
    \item curve is always a surface curve of the polyhedral surface and not violating mathematical properties. This also allows for direct rendering without projection.
    \item Useful for mesh processing
    \item Interpretation of Discretization is more consistent: surface mesh may be discretized approximationof real-world manifold with the mesh to be the finite grid. then it is only natural to discretize a real world surface curve in the same way.
  \end{itemize}

  Advantages of the triangular data structure over the edge-based one:
  \begin{itemize}
    \item Triangles can be easier generalized to higher dimensions than edges.
    \item Triangles do not need to handle boundaries in complicated ways.
    \item Triangles allow easy and ordered access to fan around vertex.
    \item Artifact removal routine for triangles is simpler than for edges.
    \item Triangles do not exhibit data loss when handling reflex vertices
    \item Triangles allow for a more general moving of points
  \end{itemize}
% subsection discrete_surface_curve_data_structure (end)

\subsection{Initial Curve Selection} % (fold)
\label{sub:initial_curve_selection}
  As for the vast majority of optimization algorithms, the results and efficient working are highly dependent on the chosen starting values.
  Curve smoothing itself is an optimization process and to solve the boundary value problem, we need an initial value.
  So, choosing the initial curve in the right way will also heavily change the speed and quality of the algorithm.
  The algorithm should be able to handle a vast amount unsmooth curves.
  Still, the handling of artifacts will be taken care of at the start.
  \paragraph{User Interface for Selecting and Controlling Initial Curves}
  \paragraph{Drawing by Ray Tracing}
  \paragraph{Connecting the Vertices}
  \paragraph{Closed Initial Curves and Fixed Vertices}
  \paragraph{Artifact Removal}
  \paragraph{Smoothed Curvature Values by Stencil}
    Stencil is discrete approximation of solution to Laplacian equation or heat equation which is smooth (infinitely differentiable).
    Maybe it would be useful to keep the length of the curve and still smooth it.
  \paragraph{Vertex Curves}
  \paragraph{Face Curves}
  \paragraph{Tracing Geodesics}
% subsection initial_curve_selection (end)

\subsection{Unfolding} % (fold)
\label{sub:unfolding}
To unfold two triangles along their common edge and find the shortest line connecting two points, we will first look back into two dimensions.

\begin{figure}[h]
  \begin{center}
    \large
    INSERT YOUR IMAGE HERE!
  \end{center}
  \caption[2D Unfolding Primitive]{%
    \textbf{2D Unfolding Primitive}
  }
\end{figure}

Let $p,q\in\setReal^2$ such that $p_x < q_x$.
\[
  \Delta x = q_x - p_x
  \separate
  \Delta y = q_y - p_y
\]
The straight line function connecting these two points then looks like the following.
\[
  f(x) = \frac{q_y - p_y}{q_x - p_x}(x - p_x) + p_y = \frac{q_y - p_y}{q_x - p_x}x + \frac{p_yq_x - q_yp_x}{q_x - p_x}
\]
Please note that this function is well-defined, as $q_x - p_x > 0$.
To get the intersection point with the ordinate, set the argument to zero.
\[
  t\define f(0) = \frac{p_yq_x - q_yp_x}{q_x - p_x}
\]
\begin{figure}[h]
  \begin{center}
    \large
    INSERT YOUR IMAGE HERE!
  \end{center}
  \caption[Unfolding of two Triangles]{%
    \textbf{Unfolding of two Triangles}
  }
\end{figure}

Applying this expression to the case of two triangles that needs to be unfolded, the following results.
\[
  e \define \frac{v_2 - v_1}{\norm{v_2 - v_1}}
  \separate
  p \define r_1 - v_1
  \separate
  q \define r_2 - v_1
\]
\[
  p_y = \scalarProduct{e}{p}
  \separate
  q_y = \scalarProduct{e}{q}
\]
\[
  p_x = -\norm{p - p_y e}
  \separate
  q_x = \norm{q - q_y e}
\]
Unfolding two adjacent triangles that share a common edge from three-dimensional space into the two-dimensional plane is not unique as there are two cases.
In one case, the triangles might overlap and in the other not.
The above sign for $p_x$ is therefore important.

\begin{lemma}[Unfolding leads to geodesic]
  Connecting $p$, $r$, and $q$ in that order by straight lines leads to locally shortest and straightest geodesics.
  It solves the discrete boundary value problem for locally shortest and straightest geodesics when boundary points lie in the inside of adjacent triangles.
\end{lemma}
\begin{proof}
  According to \textcite{polthier2006} and \textcite{martinez2005}, locally shortest and straightest geodesics coincide on polyhedral if they do not contain any vertices.
  So, it is sufficient to show that the discrete curvature at the crease is zero.
  But this has be true by construction.
\end{proof}
% subsection unfolding (end)

\subsection{Unfolding with Desired Curvature} % (fold)
\label{sub:unfolding_with_desired_curvature}
Again, a two-dimensional primitive is used.
\[
  f(x) = -\cot α (x - p_x) + p_y
\]
\[
  t = f(0) = p_y + \absolute{p_x}\cot α
\]
Now, combine two points to both sides with a desired discrete curvature.
For this, assume $p_x < 0$ and $q_x > 0$.
Furthermore, the desired curvature is positive $κ > 0$ and taken mathematically positive.
\[
  κ + π = α + β
\]
From the former computation, it is clear that their points on the ordinate have to coincide.
\[
  p_y - p_x \cot α = q_y + q_x \cot β
\]
\[
  \cot β = \cot(π-(α-κ)) = -\cot(α-κ)
\]
\[
  \cot(α-κ) = \frac{\cot α \cot κ + 1}{\cot α - \cot κ}
\]
\[
  p_y - p_x \cot α = q_y - q_x \frac{\cot α \cot κ + 1}{\cot α - \cot κ}
\]
\[
  φ\define \cot{α}
  \separate
  K\define \cot κ
\]
\[
  p_y - p_x φ = q_y - q_x \frac{Kφ + 1}{φ - K}
\]
\[
  q_x \frac{Kφ + 1}{φ - K} = q_y - p_y + p_x φ
\]
\[
  Kq_xφ + q_x = (q_y-p_y)(φ - K) + p_x φ (φ - K)
\]
\[
  Kq_xφ + q_x = \Delta y φ - K\Delta y + p_x φ^2 - Kp_x φ
\]
\[
  q_x + K\Delta y = p_x φ^2 + (\Delta y -K(p_x+q_x))φ
\]
\[
  \frac{q_x + K\Delta y}{p_x} = φ^2 + \frac{\Delta y - K\Sigma x}{p_x}φ
\]
\[
  φ = -\frac{\Delta y - K\Sigma x}{2p_x} \pm \sqrt{\frac{q_x + K\Delta y}{p_x} + \frac{(\Delta y - K\Sigma x)^2}{4p^2_x}}
\]
\[
  t = p_y - p_x φ
\]
\[
  t = p_y + \frac{\Delta y - K\Sigma x}{2} \mp \sqrt{p_x(q_x + K\Delta y) + \frac{(\Delta y - K\Sigma x)^2}{4}}
\]
\[
  t = \frac{\Sigma y - K\Sigma x}{2} \mp \sqrt{p_x(q_x + K\Delta y) + \frac{\Delta^2y - 2K\Delta y\Sigma x + K^2\Sigma^2x}{4}}
\]
\[
  t = \frac{\Sigma y - K\Sigma x}{2} \mp \sqrt{p_xq_x(1 + K^2) + \frac{\Delta^2y - 2K\Delta y\Delta x + K^2\Delta^2x}{4}}
\]
\[
  t = \frac{\Sigma y - K\Sigma x}{2} \mp \sqrt{p_xq_x(1 + K^2) + \frac{(\Delta y - K\Delta x)^2}{4}}
\]
% subsection unfolding_with_desired_curvature (end)

\subsection{Desired Curvature Stencil} % (fold)
\label{sub:desired_curvature_stencil}

% subsection desired_curvature_stencil (end)

\subsection{Desired Curvature Mapping} % (fold)
\label{sub:desired_curvature_mapping}

% subsection desired_curvature_mapping (end)

\subsection{Curve Smoothing Algorithm} % (fold)
\label{sub:curve_smoothing_algorithm}
  Curve smoothing is not a discrete problem.
  Using only a finite amount of steps is probably not possible.
  So, we use a process of convergence.
  This has the advantage that not all triangles need to be unfolded at once.
  This makes the implementation and parallelization much simpler.
  The discrete algorithms for tracing geodesics are not easy to parallelize.

  \paragraph{Idea and Overview}
  \paragraph{Edge Vertex Relaxation}
    We need to take a look at topological and numerical robustness.
  \paragraph{Vertex Vertex Relaxation}
  \paragraph{Critical Vertex Handling}
  \paragraph{Artifact Removal and Self-Intersection Handling}
  \paragraph{Desired Curvature Mapping}
    Desired curvatures need to be constant along edges.
    Therefore interpolate on angle-basis around vertex.
    We do not want to loose information of desired curvatures.
  \paragraph{Curve Evaluation}
  \paragraph{Correctness and Convergence}
    Correctness can be shown by showing the convergence to the curvature values.
    This by definition of the given curvature values smooths the curve.
    The convergence might only be shown for contracting the curve by smoothing.
    As the limit is the geodesic, the prove of its convergence is there.
  \paragraph{Complexity}
% subsection curve_smoothing_algorithm (end)

% section design (end)
\end{document}
